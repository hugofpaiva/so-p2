\documentclass[10pt,portuguese]{article}

\usepackage{fourier}

\usepackage[]{graphicx}
\usepackage[]{color}
\usepackage{xcolor}
\usepackage{alltt}
\usepackage{listings}
\usepackage[T1]{fontenc}
\usepackage[utf8]{inputenc}
\setlength{\parskip}{\smallskipamount}
\setlength{\parindent}{5ex}
\usepackage{indentfirst}
\usepackage{listings}
\usepackage{setspace}
\usepackage{hyperref}
\hypersetup{
    colorlinks=true,
    linkcolor=auburn,
    filecolor=magenta,      
    urlcolor=blue, urlsize=2em
}

% Set page margins
\usepackage[top=100pt,bottom=100pt,left=68pt,right=66pt]{geometry}

% Package used for placeholder text
\usepackage{lipsum}

% Prevents LaTeX from filling out a page to the bottom
\raggedbottom


\usepackage{fancyhdr}
\fancyhf{} 
\fancyfoot[C]{\thepage}
\renewcommand{\headrulewidth}{0pt} 
\pagestyle{fancy}

\usepackage{titlesec}
\titleformat{\chapter}
   {\normalfont\LARGE\bfseries}{\thechapter.}{1em}{}
\titlespacing{\chapter}{0pt}{50pt}{2\baselineskip}

\usepackage{float}
\floatstyle{plaintop}
\restylefloat{table}

\usepackage[tableposition=top]{caption}



\frontmatter

\definecolor{light-gray}{gray}{0.95}

\renewcommand{\contentsname}{Índice}

\begin{document}

\selectlanguage{portuguese}

\begin{titlepage}
	\clearpage\thispagestyle{empty}
	\centering
	\vspace{2cm}

	
	{\Large  Sistemas Operativos \par}
	\vspace{0.5cm}
	{\small Professor: \\
	José Nuno Panelas Nunes Lau\par}
	\vspace{4cm}
	{ \textbf{Problema genérico de gestão de recursos:}} \\
	\vspace{0.5cm}
	{\Huge \textbf{Fumadores}} \\
	\vspace{1cm}
	\vspace{4cm}
	{\normalsize Carolina Araújo, 93248 \\ 
	             Hugo Paiva, 93195
	   \par}
	   	{\tiny
	Igual distribuição de trabalho \\entre os dois membros\par}
	\vspace{2cm}

    \includegraphics[scale=0.20]{images/logo_ua.png}
    
    \vspace{2cm}
    
	{\normalsize DETI \\ 
		Universidade de Aveiro \par}
		
	{\normalsize 30-12-2019 \par}
	\vspace{2cm}
		
	
	\pagebreak

\end{titlepage}
\tableofcontents{}
\clearpage

\section{Introdução}
\par Este trabalho prático foi desenvolvido com o objetivo de compreender os mecanismos associados à execução de processos e \textit{threads}. 

\par Para empreender este propósito, foi pedido que se solucionasse um problema que envolve vários fumadores com necessidades distintas para fumar. Dito isto, implementou-se um programa em C que simula e soluciona o problema recorrendo a semáforos e a memória partilhada, de modo a sincronizar os vários processos independentes.

\clearpage

\section{Contextualização}

\subsection{Sincronização}

Explicar por alto como funcionam os semáforos e threads

\clearpage

\subsection{O Problema dos Fumadores}

Explicar o problema

\clearpage

\section{Implementação}

Explicar as implementações

\clearpage

\section{Resultados}

Straight foward

\clearpage

\section{Conclusão}

Straight foward

\clearpage

\section{Bibliografia}

\bibliographystyle{plain}

\bibliography{biblist}

\vspace{5mm} %5mm vertical space

[1] \url{O Deus LAU}


\end{document}

